\documentclass{scrartcl}
\usepackage{polyglossia}
\setdefaultlanguage{english}
\usepackage{xltxtra}
\setmainfont[Mapping=tex-text]{Junicode}
\newfontfamily{\avestan}[Scale=0.9]{Junicode}
\usepackage[letterpaper,margin=1in,includehead,includefoot]{geometry}
\usepackage[object=vectorian]{pgfornament}
% MISC packages
\usepackage{hanging,verbatim,fancyhdr,xcolor,calc,multirow,longtable}
\usepackage[hidelinks]{hyperref}
\hypersetup{% 
pdftitle={Curriculum Vitae},%
pdfauthor={Suhas Mahesh}}
  \definecolor{gray}{HTML}{4D4D4D}
  \definecolor{white}{RGB}{255,255,255}
\pagestyle{fancy}
  \fancyhf{}
  \cfoot{Ollett CV\ \ {\footnotesize\fontspec{Printers Ornaments One}{\textcolor{gray}e}}\ \ \thepage}
  \renewcommand{\headrulewidth}{0pt}
\frenchspacing
\pagestyle{fancy}
\begin{document}
\setlength{\parindent}{0pt}

\newcommand{\sectionline}{%
  \begin{center}
  {
    \resizebox{0.8\linewidth}{2ex}
    {{%
    {\color{gray}\begin{tikzpicture}
    \node  (C) at (0,0) {};
    \node (D) at (9,0) {};
    \path (C) to [ornament=88] (D);
    \end{tikzpicture}}}}}%
    \end{center}\bigskip
  }
  
% 
\definecolor{fillcolor}{RGB}{9,97,146}

\begin{tikzpicture}[remember picture,overlay]
  % Shift the node down to create white space at the top
  \node [rounded corners=10mm, rectangle, draw=fillcolor, fill=fillcolor, anchor=north, minimum width=\paperwidth-0.5cm, minimum height=4.5cm] (box) at ([yshift=-0.25cm]current page.north){};

  \node [anchor=center] (name) at (box) {%
    \fontsize{40pt}{72pt}\color{white}%
    \textsc{Suhas Mahesh{\kern0pt}}
  };
\end{tikzpicture}


\vspace{1.5cm}
\newcommand{\pgfdot}{\hspace{1em}\tikz\draw[fill=black] (0,0) circle (.5ex);\hspace{1em}}


\vspace{-0.8cm}
\begin{center} 
{{\small 
\begin{tabular}{ccccc}
  \multirow{2}{*}{+1-647-657-4415} & \multirow{2}{*}{\pgfdot} & 
  \multirow{2}{*}{suhas.mahesh@utoronto.ca} & \multirow{2}{*}{\pgfdot} &  \multirow{2}{*}{www.suhasmahesh.com}
\end{tabular} }} \end{center}\par\vspace{1cm}

\newcommand{\secttitle}[1]{{{\Large\textbf{#1}}}\par\medskip}
\newcommand{\sectsubtitle}[1]{{{\large\textbf{#1}}}\par\medskip}

\setlength{\tabcolsep}{0pt}
\newenvironment{entrylist}{%
  \begin{longtable}{@{\extracolsep{\fill}}ll}
}{%
  \end{longtable}
}
\newcommand{\entry}[4]{%
  {\addfontfeature{Color=gray} #1}&\parbox[t]{0.8\textwidth}{%
    \textbf{#2}\hfill #3\\ #4\vspace{\parsep}%
  }\\}
\newcommand{\refentry}[3]{%
  {\addfontfeature{Color=gray} #1}&\parbox[t]{0.8\textwidth}{%
    \raggedright#2\\ {{\raggedright\small #3}}\vspace{\parsep}%
  }\\}
\newcommand{\singleentry}[2]{%
  {\addfontfeature{Color=gray} #1}&\parbox[t]{0.8\textwidth}{%
    #2\vspace{\parsep}%
  }\\}
\raggedright

\secttitle{Employment}
\begin{entrylist}
\entry
{Sep 2021–}
{Schmidt Science Fellow}
{University of Toronto}
{Department of Electrical \& Computer Engineering}
\end{entrylist}

\secttitle{Education}

\begin{entrylist}
\entry
{2016--2021}
{Doctor of Philosophy in Condensed Matter Physics}
{University of Oxford}
{Rhodes Scholarship\\
Thesis title: ``Optical and Electronic Properties of Novel Materials for Multijunction Perovskite Photovoltaics''\\
\url{http://academiccommons.columbia.edu/catalog/ac:190829}.}

\entry
{2012--2016}
{Bachelor of Science in Physics}
{Indian Institute of Science}
{With highest honors}

\end{entrylist}\bigskip\bigskip

\secttitle{Publications}



\sectsubtitle{Published articles and chapters}
\begin{entrylist}
\refentry
{2023}{*“A Mirror and a Handlamp”: \emph{The Way of the Poet-King} and the Afterlife of the \emph{Mirror} in the World of Kannada Literature (edited by me, with contributions by Sarah Pierce Taylor, Gil Ben-Herut, and myself)}{Chapter 3 in Yigal Bronner (ed.), \emph{A Lasting Vision: Daṇḍin’s Mirror in the World of Asian Literature}, 92–140. Delhi: Oxford University Press. DOI \href{https://doi.org/10.1093/oso/9780197642924.003.0003}{\texttt{10.1093/oso/9780197642924.003.0003}}.}
\end{entrylist}\bigskip

\sectsubtitle{Published books}
\begin{entrylist}
\singleentry
{2021}{\emph{Lilavai} (edition and translation)\\
Cambridge, Mass.: Harvard University Press (Murty Classical Library of India)\\
\url{https://www.hup.harvard.edu/catalog.php?isbn=9780674247598}\\
ISBN 9780674247598 · 432 pages}
\end{entrylist}

\sectsubtitle{Manuscripts in press}
\begin{entrylist}
\refentry
{\emph{forthcoming}}{\emph{The Mirror of Ornaments} / \emph{Alaṅkāradappaṇō}.}{to be published by Unior Press, Naples}
\end{entrylist}

\sectsubtitle{Works in progress}
\begin{entrylist}
\refentry
{\emph{under contract}}{\emph{The Plays of Māyurāja} (edition and translation, with Naresh Keerthi)}
{Cambridge, Mass.: Harvard University Press (Murty Classical Library of India)}
\end{entrylist}

\sectionline

\sectsubtitle{Edited issues}
\begin{entrylist}
\singleentry
{2018}{Special issue on the Sātavāhanas in the \emph{Journal of the International Association of Buddhist Studies} 41 (appeared in 2019), with papers by Alice Collett, David Efurd, Andrew Ollett, Akira Shimada, Meera Visvanathan, and Monika Zin.}
\end{entrylist}

\sectsubtitle{Reviews}
\begin{entrylist}
\refentry
{2018}{Review of Johannes Bronkhorst, \emph{A Śabda Reader: Language in Classical Indian Thought}}
{\emph{Philosophy East and West} 71.2 (2021): 1–5. DOI \href{https://doi.org/10.1353/pew.2021.0036}{\texttt{10.1353/pew.2021.0036}}.}
\end{entrylist}

\sectionline


\secttitle{Grants, Fellowships and Prizes}
\begin{entrylist}
\refentry
{2021–2024}{Entanglements of the Indian Past}
{Neubauer Collegium Faculty Research Project (with Whitney Cox, Dipesh Chakrabarty, Sarah Pierce Taylor, and Anand Venkatkrishnan)\\
\url{https://neubauercollegium.uchicago.edu/research/entanglements-of-the-indian-past}}
\end{entrylist} 

\sectionline

\secttitle{Conferences, workshops, and lectures}
(Only unpublished papers are listed here)
\begin{entrylist}
\refentry
{July 2023}{Theorizing Sarcasm}
{Twelfth Coffee Break Conference, Groningen}
\end{entrylist}

\sectionline

\secttitle{Teaching}

\textbf{at the University of Chicago}:\vspace{1ex}

\quad\begin{minipage}{0.9\textwidth}
Language and reading courses:\vspace{1ex}

\quad\begin{minipage}{0.9\textwidth}
First-year Sanskrit (AY 2017–2018, AQ 2020, 2021, WQ 2022)\\[.5ex]
Advanced Sanskrit (AQ 2017, SQ 2020, AQ 2021)\\[.5ex]
Introduction to Prakrit (SQ 2018)\\[.5ex]
Directed readings in Sanskrit and/or Prakrit (WQ \& SQ 2020, WQ \& SQ 2022)\\
\end{minipage}
\end{minipage}\vspace{-.75ex}

\quad\begin{minipage}{0.9\textwidth}
Content courses:\vspace{1ex}

\quad\begin{minipage}{0.9\textwidth}
Classical Literature of South Asia, Part 1 (AQ 2019, 2021)\\[.5ex]
Introduction to Indian Philosophy, Part 2 (SQ 2021) [with Anand Venkatkrishnan]\\[.5ex]
Theater of Premodern South Asia (SQ 2020)
\end{minipage}
\end{minipage}\medskip

\quad\begin{minipage}{0.9\textwidth}
College core:\vspace{1ex}

\quad\begin{minipage}{0.9\textwidth}
Language and the Human (WQ 2021, 2022)\\[.5ex]
Poetry and the Human (WQ 2020)\\[.5ex]
Readings in World Literature (WQ 2018)
\end{minipage}
\end{minipage}\medskip\medskip

\textbf{outside the University of Chicago}:

\begin{entrylist}
\singleentry
{July 2017}{(with Eva de Clercq) led an ‘Apabhramsha Reading Retreat’ in Castelldefels}
\singleentry
{Spring 2015}{Instructor, Introduction to Indian Philosophy, Columbia University}
\singleentry
{Fall 2012}{Instructor, Intermediate Sanskrit, Columbia University}
\singleentry
{Fall 2012}{Teaching Assistant, Advanced Sanskrit, Columbia University}
%\singleentry
%{2008–2012}{Private tutoring in Ancient Greek and Sanskrit}
\end{entrylist} 

\secttitle{Graduate students advised}

\hangpara{0.15in}{1}Anil Mundra (UChicago, History of Religions, graduated Summer 2022; currently a postdoctoral fellow at Rutgers)\bigskip

\secttitle{Editorship}

\hangpara{0.15in}{1}Co-founder and editor, \emph{New Explorations in South Asia Research} (\href{https://nesarjournal.org}{NESAR}), ISSN 2834-3875, 2022–present.\smallskip

\vfill


\end{document}
